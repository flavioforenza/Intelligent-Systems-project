\section*{INTRODUCTION}
In recent years, the field of computer vision is becoming more and more important 
due to its various applications exercised in different areas such as 
intelligent surveillance and monitoring, recognition, self-driving cars, warehouse 
management, health and medicine, sports, drones and robotics. The 
goal of computer vision is to allow the computer to see, identify and process 
images, or videos, as it happens in human vision, providing adequate output. 
It is a multidisciplinary field that could be considered as a subfield of 
Artificial Intelligence (AI) and Machine Learning (ML). Computer vision is 
shifting from statistical methods to deep learning neural networks methods. 
Deep Learning (DL) is used in the domain of digital image processing to solve 
difficult problems. DL is largely based on Artificial Neural Networks (ANNs), 
a paradigm inspired by the functioning of the human brain. Just like the latter, 
it is made up of many computing cells, called \emph{neurons}, whose task 
is to carry out operations and interact with each other to make a decision. 
DL methods, such as Convolutional Neural Networks (CNNs, or ConvNet), 
improve prediction performance by using datasets and abundant computing 
resources. Thanks to their integration, it was possible to carry out visual 
recognition activities, based on the classification of images, localization, detection 
and semantic segmentation, all thanks to the use of kernels (also 
known as filters), which go to extract features useful in a given context. This 
has made CNN networks the main topic of deep learning algorithms in computer 
vision. Compression techniques based on the use of CNN networks and 
computer vision are widespread. These methods have the task of preserving 
space on existing media through the use of encodings with or without loss of 
quality. Object detection, for computer vision, is also one of the key factors 
in understanding the scene. In this regard, many segmentation algorithms 
have been proposed which aim to separate the foreground objects from the 
background. As with objects, recognition of a subject can also take place 
in multiple ways. Among these methods appear those that have the task 
of recognizing the gait of an individual and predicting information relating 
to identity, age and gender. Face recognition appears to be a difficult challenge 
when they appear to be similar. Also for this purpose, some methods 
have been designed to have models capable of distinguishing similar subjects 
from less similar ones. To date, the challenges that we try to overcome in 
computer vision, but especially in convolutional networks, are those related 
precisely to the images, their availability, useful for training the models, their 
quality and the possible associated labels. Unfortunately, there are still very 
few datasets with images useful for a particular purpose and many of their 
images appear to be of low quality. Multiple solutions have been found for 
this problem, one of the simplest, but expensive, is to use equipment such as 
good stereo cameras, high resolution cameras and tripods. Another solution 
to this problem is to use algorithms capable of improving images through the 
use of noise reduction or color costancy research techniques. The following 
paper is divided into ten sections where in each of these new methodologies 
are explained compared with those already existing in the state of the art, 
used in the field of computer vision. In order of presentation, the topics 
covered are: \emph{Salient structure, Superpixel, Compression, Color Constancy, 
Gait Recognition, Semantic Segmentation, Noise Removal, Tracking, Face 
Recognition and Identification}.
