\documentclass[letterpaper,12pt]{article}
\usepackage{cite}
\usepackage{graphicx}
\usepackage{amsmath}
\usepackage{adjustbox}
\usepackage{multirow}
\usepackage{caption}
\usepackage{subcaption}
\newcommand{\RN}[1]{%
  \textup{\uppercase\expandafter{\romannumeral#1}}%
}
\captionsetup[figure]{labelsep=period}
\captionsetup[subfigure]{labelformat=simple} % default is 'parens'
\renewcommand\thesubfigure{\thefigure.\alph{subfigure}.}
\DeclareMathOperator*{\argmax}{argmax}
\DeclareMathOperator*{\argmin}{argmin}
\newtheorem{corollary}{Corollary}

\begin{document}
    
\title{\bfseries{Computer Vision and CNNs}}
\author{Flavio Forenza}
\date\today
\maketitle

\begin{figure}[h!]
  \centering
  \includegraphics[width=0.2\linewidth]{images/logoUnimi2.png}
  \centering
\end{figure}

\begin{abstract}
    This report contains a summary of some proposed state-of-the-art 
    methodologies relating to Computer Vision and Convolutional Neural 
    Networks (CNN).
\end{abstract}

\section*{INTRODUCTION}
In recent years, the field of computer vision is becoming more and more important 
due to its various applications exercised in different areas such as 
intelligent surveillance and monitoring, recognition, self-driving cars, warehouse 
management, health and medicine, sports, drones and robotics. The 
goal of computer vision is to allow the computer to see, identify and process 
images, or videos, as it happens in human vision, providing adequate output. 
It is a multidisciplinary field that could be considered as a subfield of 
Artificial Intelligence (AI) and Machine Learning (ML). Computer vision is 
shifting from statistical methods to deep learning neural networks methods. 
Deep Learning (DL) is used in the domain of digital image processing to solve 
difficult problems. DL is largely based on Artificial Neural Networks (ANNs), 
a paradigm inspired by the functioning of the human brain. Just like the latter, 
it is made up of many computing cells, called \emph{neurons}, whose task 
is to carry out operations and interact with each other to make a decision. 
DL methods, such as Convolutional Neural Networks (CNNs, or ConvNet), 
improve prediction performance by using datasets and abundant computing 
resources. Thanks to their integration, it was possible to carry out visual 
recognition activities, based on the classification of images, localization, detection 
and semantic segmentation, all thanks to the use of kernels (also 
known as filters), which go to extract features useful in a given context. This 
has made CNN networks the main topic of deep learning algorithms in computer 
vision. Compression techniques based on the use of CNN networks and 
computer vision are widespread. These methods have the task of preserving 
space on existing media through the use of encodings with or without loss of 
quality. Object detection, for computer vision, is also one of the key factors 
in understanding the scene. In this regard, many segmentation algorithms 
have been proposed which aim to separate the foreground objects from the 
background. As with objects, recognition of a subject can also take place 
in multiple ways. Among these methods appear those that have the task 
of recognizing the gait of an individual and predicting information relating 
to identity, age and gender. Face recognition appears to be a difficult challenge 
when they appear to be similar. Also for this purpose, some methods 
have been designed to have models capable of distinguishing similar subjects 
from less similar ones. To date, the challenges that we try to overcome in 
computer vision, but especially in convolutional networks, are those related 
precisely to the images, their availability, useful for training the models, their 
quality and the possible associated labels. Unfortunately, there are still very 
few datasets with images useful for a particular purpose and many of their 
images appear to be of low quality. Multiple solutions have been found for 
this problem, one of the simplest, but expensive, is to use equipment such as 
good stereo cameras, high resolution cameras and tripods. Another solution 
to this problem is to use algorithms capable of improving images through the 
use of noise reduction or color costancy research techniques. The following 
paper is divided into ten sections where in each of these new methodologies 
are explained compared with those already existing in the state of the art, 
used in the field of computer vision. In order of presentation, the topics 
covered are: \emph{Salient structure, Superpixel, Compression, Color Constancy, 
Gait Recognition, Semantic Segmentation, Noise Removal, Tracking, Face 
Recognition and Identification}.


%\newpage
%\section{A Unified Framework for Salient Structure Detection by Contour-Guided Visual Search}

\begin{flushright}
    \author{
    Kai Fu Yang,
    Hui Li,
    Chao-Yi Li,
    and Young-Jie Li,
   \emph{Member}, 
    IEEE 
}
\end{flushright}

\begin{center}
    \emph{IEEE TRANSACTIONS ON IMAGE PROCESSING, VOL. 25, NO. 8, AUGUST 2016}
\end{center}

\subsection{INTRODUCTION}
In order to reduce the complexity in the analysis of a scene, a useful method 
is introduced to detect potential information, such as regions or 
objects, simultaneously. This method is called "\emph{Visual Saliency}". Before introducing
the topic, four types of concepts must be explained: (1)\emph{Fixations}: 
they concern the scene framed by the human eye. They are used to compare 
the methods of forecasting fixations. (2)\emph{ROI}: each region contains information, 
such as light or dark objects, which is intended to be separated.(3)\emph{Salient objects}: 
animals, people, cars etc. (4)\emph{Salient edges}: boundary of each object. The proposed 
method is based on carrying out a \emph{Salient Structure (SS) Detection},
useful for identifying the four previous properties, both in cluttered and 
simple scenes. The proposed framework is called \emph{CGVS (contour-guided visual 
search)}. This visual search tool identifies the targets using two types of 
paths: (1)\emph{selective path}: the boundaries of each object are detected, useful 
for estimating the position and size of the ROI: (2)\emph{non-selective path}: it 
can be properties such as color, luminary, texture etc. This search strategy 
carries out parallel processing on both paths, extracting global and local 
information. Finally, a Bayesian inference is applied to integrate the contour 
based spatial prior (CBSP), a useful method for extracting information on 
the boundaries, and local information in order to identify the salience of each 
pixel. 

\subsection{RELATED WORK}

\subsubsection{\emph{Fixation Prediction}}
Fixation prediction models aim to calculate salience maps, which are used to 
identify ROIs. Fixation prediction models provide smooth regions of interest 
rather than uniform regions that highlight all salient objects. These models 
only offer the ability to detect the position of potential objects while excluding 
other features such as surfaces or shapes, which are useful for object 
detection and recognition. The methods proposed to obtain a salience map 
are varied and range from the use of Bayesian Framework, to including those 
for machine learning. 

\subsubsection{\emph{Salient Object Detection}}
Existing methods, useful for extracting salient objects from a scene, rely on 
the contrast of the local or global region. Object detection is an operation 
related to the "object proposal" operation which attempts to generate a set 
of all objects in the scene, regardless of their salience. In this paper Bayesian 
inference is used to accomplish this task.

\subsubsection{\emph{Bridging the Two Tasks}}
Unlike the existing methods, the proposed one is able to obtain more detailed 
salient structures in terms of resolution, moreover it is able to operate in 
conditions in which the scene is simple or complex. Both tasks do not need 
specific tuning.

\subsection{CONTOUR-GUIDED VISUAL SEARCH MODEL}

\begin{figure}[htbp]
    \centering
    \includegraphics[width = 0.8\linewidth]{images/selective and non-selective pathways.png}
    \centering
    \caption{The flowchart of the porposed system}
    \label{fid: flowchart}
\end{figure}

The research of the possible potential positions, of the salient structures, is 
carried out with the help of the contour-based spatial prior (CBSP) process, 
which uses the contours detected in a non-selective path. At the same time, 
in the selective path, the local characteristics concerning color, luminance 
and texture are extracted. The information obtained from the CBSP is inserted 
into the Bayesian framework for the prediction of the salient structure. 
Finally, the output of the Bayesian framework will represent the salient structure 
which will be improved through an iterative procedure applied on the 
CBSP process. The flowchart of the proposed method is summarized in Fig. \ref{fid: flowchart}.

\subsubsection{\emph{Contour-Based Spatial Prior (CBSP)}}
\begin{figure}[htbp]
    \centering
    \includegraphics[width = 0.8\linewidth]{images/CBSP.png}
    \centering
    \caption{CBSP reconstruction}
    \label{fid: CBSP}
\end{figure}
Within the non-selective path, the crucial operation is to calculate the 
approximate spatial weights of salience based on the distribution of the previously 
detected contours. In this path there is the problem of determining 
the local information of each object. The solution to the problem is based 
on an operation that fills the regions delimited by these contours. A very 
efficient method proposed in \cite{0747815500} result is used for the extraction of the contours.
An example of CBSP reconstruction is shown in Fig. \ref{fid: CBSP}. Each pixel 
has a direction represented by a radius $ d_r $ which is equal to $d_r = min (H, W) / 3$, where 
\emph{H} and \emph{W} correspondingly indicate the width and height of the entire image.
The diameter divides the circle into two parts where, on each of these, for 
each pixel belonging to the edge, the average response of the edge (AER) will 
be calculated. This calculation is useful in deciding which of the two halves 
will belong to the salient region. Each spatial weight of salience is denoted 
by $ S_e $, while the added bias is denoted by $ S_c $ and both are normalized in the 
range of [0,1]. The value, which in turn will be normalized, of CBSP is given by: $$ S_w = S_e + S_c $$

\subsubsection{Low-Level Feature Extraction}
Features such as color (rgb), luminance (flum) and texture are extracted in 
parallel. There are complicated regions in a texture that can be represented. 
For this purpose, the texture channel ($ {f_{ed}} $)  has been introduced, which is 
useful to better represent the salient regions using the contour density (ED). 
The texture channel is calculated by using an 11x11 filter that smooths the 
edges.

\subsubsection{Bayesian Inference With Contour Guidance}
Bayesian inference gives the possibility to calculate the probability that a 
pixel x may belong to a salient structure s. This probability is computed as:
$$ p(s|x) = \frac{p(s)p(x|s)}{p(s)p(x|s)+p(b)p(x|b) } $$
Where $ p(x|s) $ is the probability of a pixel belonging to a salient structure, 
$ p(x|b) $ is the probability that a pixel belongs to the background, $ p(b)=1-p(s) $,
finally $ p(s)=Sw $ represents the CBSP value. The implementation 
details are as follows.
\begin{enumerate}
    \item \emph{Predict the size of Potential Structure}: this procedure, through a binarization of the map obtained 
    from $ p(s) $, separates the salient structures from the background, all 
    through the use of a threshold. This threshold is calculated using the 
    local parameters (color, luminance, etc.), a percentage not greater than 
    50\% useful to denote the size of the salient structure, an initial weight 
    and finally two sets of pixels corresponding to the background and the 
    foreground (salient structure).
    \item \emph{Evaluate the Importance of Each Feature}: a feature is considered important 
    when the difference between the average pixel values of its 
    salient structure and those of the background is large enough, i.e. when \emph{$ w_i $} 
    takes on a satisfiable value.
    \item \emph{Calculate the Observation Likelihood}:in order to calculate the observed 
    probabilities, the different color channels \emph{($ f_{lum}, f_{rg}, f_{by}, f_{ed} $)} are used, 
    and the distribution functions $ p(x_i | S_{T_{opt}}) $ and $ p(x_i | B_{T_{opt}}) $ where $ S_{T_{opt}} $ 
    represents the set of pixels belonging to the structure salient, while 
    $ B_{T_{opt}} $ represents the set of pixels belonging to the background.
    $$ p(x|s) = \prod_{i \in {\{} f_{lum}, f_{rg}, f_{by}, f_{ed} {\}}} {(p(x_i | S_{T_{opt}}))^{w_i}} $$
    $$ p(x|b) = \prod_{i \in {\{} f_{lum}, f_{rg}, f_{by}, f_{ed} {\}}} {(p(x_i | B_{T_{opt}}))^{w_i}} $$
    \item \emph{Enhance the Salient Structure by Iterating}: the salient structures, at 
    each instant \emph{t}, with $ t = t_0,…, t_n $, are iteratively improved by applying 
    a median filter on each of them. In addition, the weight attributed to 
    the importance of each feature is changed. Finally, the CGVS model 
    will operate on a set of instants \emph{t}, thus forming the contour-guided 
    visual search.
\end{enumerate}

\subsection{EXPERIMENTS}
The proposed method was evaluated on different datasets. For the fixation 
prediction the datasets are \cite{0747815530} \cite{0747815531} \cite{0747815579}, while for the salient object detection \cite{0747815506} \cite{0747815508} \cite{0747815518}.

\subsubsection{\emph{Basic Property of the Proposed Model}}
The proposed system allows to carry out a multiple search of objects. In 
comparison to the work of Itti et al. \cite{0747815505} the proposed system is able to detect, 
after several iterations, the region and therefore the salient object.

\subsubsection{\emph{Fixation Prediction}}
Several proposed methods, such as the CA, in order to increase the ROC 
index, aim to have a first fixation that is limited to having a central prior 
in the scene. On the other hand, this method will lose information on surrounding 
objects. In the proposed method, the computation of the CBSP 
takes into account this center prior, without however losing the information 
on the other objects in the scene.




%\newpage
%\section{Linear Spectral Clustering Superpixel}

\begin{flushleft}
    \author{
    Jiansheng Chen, 
    \emph{Member, IEEE}, 
    Zhengqin Li,
    \emph{Student Member, IEEE}, 
    Bo Huang 
}
\end{flushleft}

\begin{center}
    \emph{IEEE TRANSACTIONS ON IMAGE PROCESSING, VOL. 26, NO. 7, JULY 2017}
\end{center}

\subsection{INTRODUCTION}
The introduced technique is called SUPERPIXEL. Widely used in image 
processing for particular tasks such as image segmentation, image analysis, 
image classification, target tracking, 3D reconstruction, surface retrieval and 
object proposal. The purpose of this technique is to be able to group the 
pixels into groups that delimit the edges of an object in order to be able 
to extract its content. Compared to other methods already existing in the 
state of the art, characteristics such as size, number of superpixels and shape 
are not considered. The purpose of the elaborate story is to reduce the 
computational complexity. The three targets that must be satisfied by each 
superpixel algorithm are:
\begin{enumerate}
    \item Adhere well to the edges without forming overlaps on objects;
    \item Have a pre-processing technique useful to improve efficiency;
    \item Consider global information.
\end{enumerate}
The proposed system, called Linear Spectral Clustering (LSC), manages to 
satisfy all the previous points with a high memory efficiency. In LSC, each 
pixel is mapped to a point within a ten-dimensional space of characteristics 
in which the weighted K-means is applied for segmentation. 

\subsection{LSC SUPERPIXEL}
The study focuses on the relationships between the results returned by the 
optimization functions, and those returned by two other equations. If the results 
are equivalent, then weighted k-means clustering can replace the highly 
complex eigen based method.

\subsubsection{Mathematical Backgrounds}
The task of LSC is to discover the relationships that exist between the objective 
functions ($ F_{N_{cuts}} $) (\ref{FNcuts}), of the normalized cuts, and the objective functions 
of the weighted K-means ($ F_{km}$) (\ref{Fkm}). 
\begin{equation} \label{Fkm}
    F_{km} = \sum_{k=1}^K\sum_{p\in\pi_k}\omega(p)= || \phi(p)-m_k ||^2 
\end{equation}
\begin{equation} \label{Centroid}
    m_k = \frac{\sum_{q\in\pi_k}\omega(q)\phi(q)}{\sum_{q\in\pi_k}\omega(q)}
\end{equation}
\begin{equation} \label{FNcuts}
    F_{N_{cuts}} = \frac{1}{K}\sum_{k=1}^K\frac{\sum_{p\in\pi_k}\sum_{q\in\pi_k}W(p,q)}{\sum_{p\in\pi_k}\sum_{q\in{V}}W(p,q)}
\end{equation}
Where $ m_k $ (\ref{Centroid}) represent the average point, or centroid, of each cluster. Within 
the formulation of normalized cuts, each data Point corresponds to a node 
within a graph \emph{G = (V, E, W)} where \emph{V} is the set of all nodes, \emph{E} represents 
the set of all edges connected and \emph{W} represents the result returned by a 
similarity function between nodes. The criterion of normalized cuts is based 
on maximizing the $ F_{N_{cuts}} $ function. The minimization or maximization problems 
are called \emph{optimization problems} which can be solved by using a positive kernel of 
a matrix. To see the relationship between the two functions, the Dhillon's concept \cite{0781426526} is extended, to obtain {\bfseries Corollary \ref{Corollary1}}:

\begin{corollary} \label{Corollary1}
    Optimizations of the objective functions of the weighted K-means
    and the normalized cuts are mathematically equivalent if (\ref{eq4}) and (\ref{eq5}) 
    hold simultaneously. The symbol $ \cdot $ stands for inner product.
\end{corollary}

\begin{equation} \label{eq4}
    \omega(p)\pi(p) \cdot \omega(q)\pi(q) = W(p,q), \forall p,q \in V
\end{equation}

\begin{equation} \label{eq5}
    w(p) = \sum_{q \in V} W(p,q), \forall p \in V
\end{equation}

After making various derivative calculations, $ F_{km} $ can be seen as: 
\begin{equation}
    F_{km} = C - K * F_{N_{cuts}} 
\end{equation}

From the above definition it is possible to note that the minimization of 
$ F_{km} $ is equivalent to the maximization of $ F_{N_{cuts}} $. Specifically, both $ F_{km} $ and 
$ F_{km} $ perform an identical partitioning of the n-dimensional space defined by 
the function $ \phi $.

\subsubsection{LSC Algorithm}
The purpose of LSC is to find the correct positive function W (p, q) in order 
to satisfy {\emph{Corollary \ref{Corollary1}}}. To achieve this, the Euclidean distance calculation 
was used as an index of similarity between two pixels. Each pixel \emph{p} is represented 
by five dimensional vectors \emph{(l, a, b, x, y)} in the CIELAB color space, where \emph{l} 
epresents brightness, \emph{a} and \emph{b} represent opposite colors and \emph{x} and \emph{y} represent
the coordinates of the plane. The similarity function, in order to satisfy 
the positivity condition required by (\ref{eq4}), is extended as a convergent Fourier series: 
\begin{equation}
    \begin{split}
        W(p,q) = C_s^2(\cos \frac{\pi}{2}(x_p-x_q)+\cos\frac{\pi}{2}(y_p-y_q)) \\
        + C_c^2(\cos \frac{\pi}{2}(l_p-l_q)+\cos\frac{\pi}{2}(\alpha_p-\alpha_q) \\
        + \cos\frac{\pi}{2}(\beta_p-\beta_q)x2.55^2)
    \end{split}
\end{equation}
Where $ C_s $ and $ C_c $ are used to control the relative significance of
color and spatial information. The mapping function that maps a point in a ten-dimensional
space is as follows:
\begin{equation}
    \begin{split}
        \phi(p) = \frac{1}{\omega(p)}[C_c\cos\frac{\pi}{2}l_p, C_c\sin\frac{\pi}{2}l_p, 2.55C_c\cos\frac{\pi}{2}\alpha_p \\
        x 2.55C_c\sin\frac{\pi}{2}\alpha_p, 2.55C_c\cos\frac{\pi}{2}\beta_p, 2.55C_c\sin\frac{\pi}{2}\alpha_p, \\
        x C_s\cos\frac{\pi}{2}x_p, C_s\sin\frac{\pi}{2}x_p, C_s\cos\frac{\pi}{2}y_p, C_s\sin\frac{\pi}{2}x_p]
    \end{split}
\end{equation}
In the defined space, the weighted K-mean clustering as well as being 
equivalent to the Fncuts optimization function, creates the optimal context to 
optimize it. The LCS algorithm takes two parameters as input; The image 
to be segmented and the preferred K number of superpixels that the 
algorithm will have to produce. Each \emph{K} pixel represents the central search point 
and its vector will be used as the weighted initial vector of the corresponding 
cluster. Each pixel is assigned to te cluster for which the weightedd mean 
is closet to the picel's vector in the ten-dimensional feature space. At each 
assignment, the weighted average and the central point will be updated until 
the system convergence. Each cluster will form a superpixel and each of 
these, if considered spatially small, can be joined to other clusters to form 
a larger cluster. The total complexity achieved by LSC is \emph{O(kN + nZ)}, where 
\emph{k} represents the number of iterations, \emph{N} the number of pixels, \emph{n} the average 
number of adjacent neighbors and \emph{z} represents the number of small 
isolated superpixels to be joined. This complexity, when compared with that 
achieved by other superpixel systems, is the lowest.

\subsection{COMPARATIVE EXPERIMENTS}
In order to evaluate the quantitative goodness of the proposed algorithm, 
three comparison metrics are used: under segmentation error (\emph{UE}) \cite{0781426514}, boundary 
recall (\emph{BR}) and achievable segmentation accuracy (\emph{ASA}). The 
goal is to have a low UE value, while it is preferable to have high values in BR 
and ASA. A low percentage of UE indicates that the detected boundary is made 
up of a small amount of pixels not belonging to the object to be delimited. 
A segmentation is correct when at least two boundary pixel fall from at least 
one superpixel boundary point. A high BR index indicates that the better 
segmentation is obtained. On the other hand, the ASA metric indicates the 
level of accurancy achieved in the segmentation pashe of contour, obtained 
thanks to the various labels (ground-truth) placed on each superpixel. A 
high ASA value indicates that superpixel adapt well to objects. The experiments 
were conducted on a set of 300 images belonging to the Barkeley 
Segmentation Database \cite{0781426515}.

\subsubsection{Parameter Selection}
When the ratio $ r = C_s / C_c $ is high, then the neighboring pixels tend to be 
aggregated into a single cluster, forming superpixels with a regular shape 
and therefore potentially incorrect. On the other hand, when \emph{r} is small, 
then it means that the pixels tend to have similar colors, therefore both 
will be grouped into a single cluster, forming irregular superpixels. In order to 
choose an optimal r value, it is necessary to introduce a metric that calculates 
the average  values of shape regularity. The average value of shape regularity 
is measured with a metric called superpixel compactness \emph{(CP)} \cite{0781426533} which is 
inversely proportional to the BR index. So as CP increases, the likelihood 
of regular shapes being created is greater and at the same time worse segmentation 
will be generated as BR decreases. The r value will be chosen 
when the value resulting from $ (1-BR / CP) $ is the lowest. Therefore \emph{r} can be 
used as a correction parameter. In order to control the search range of the 
K-means cluster algorithm, a $ \tau $ parameter is used which is equal to at least 
0.5. This value will be multiplied by the spatial components, vertical ($ v_x $) 
and horizontal ($ v_y $), of the vector belonging to each point in space. Clearly, 
as this parameter increases, the size of each superpixel generated and the 
general $ O(N) $ complexity will increase. In the experiments conducted, $ \tau $ is 
set to 1.

\subsubsection{Comparison With State-of-the-Art}
The proposed system is compared with other existing superpixels algorithms 
in the state of the art. A comparison, in terms of boundary adherence and 
speed in segmentation (Table \ref{table superpixels}), when the number of superpixels \emph{K} generated 
is 400. With a relatively high \emph{K} number, LSC achieves the best 
performance.

\begin{table}[h!]
    \centering
    \begin{adjustbox}{max width=\textwidth}
    \begin{tabular}{*{9}{|c}|}%%{|c|c|c|c|c|c|c|c|c|}
        \hline
        & EneOpt0 & SEEDS & ERS & Lattices & NCuts & SLIC & Turbo & LSC \\
        \hline
        \bfseries{ADERENCE TO BOUNDARY} & & & & & & & & \\
        \emph{Under segmentation error} & 0.230 & 0.197 & 0.198 & 0.303 & 0.220 & 0.213 & 0.277 & \bfseries{0.190}\\
        \emph{Boundary recall} & 0.765 & 0.918 & 0.920 & 0.811 & 0.789 & 0.837 & 0.739 & \bfseries{0.926}\\
        \emph{Achievable segmentation accuracy} & 0.950 & 0.960 & 0.959 & 0.933 & 0.956 & 0.956 & 0.943 & \bfseries{0.962}\\
        \hline
        \bfseries{SEGMENTATION SPEED} & & & & & & & & \\
        \emph{Computational complexity} & $ O(N^3/K^2) $ & $ O(N) $ & $ O(N^2 \lg{N}) $ & $ O(N^{\frac{3}{2}} \lg{N}) $ & $ O(N^{\frac{3}{2}}) $ & $ O(N) $ & $ O(N) $ & $ O(N) $\\
        \emph{Average time per image} & 3.35s & \bfseries{0.0935}s & 0.969s & 0.284s & 93.4s & 0.125s & 6.61s & 0.334s\\
        \hline
    \end{tabular}
    \end{adjustbox}
    \caption{Performance metrics superpixel segmentation algorithms at K=400}
    \label{table superpixels}
\end{table}
As we can see from the results, the worst algorithm, in terms of time, 
is \emph{NCuts}, while a good algorithm, which ranks second before \emph{LSC}, is \emph{SLIC}. 
Unlike the algorithm studied, SLIC uses an iterative K-means clustering 
performed inside different feature spaces, based only on local characteristics. 
On the other hand, LSC is able to use local and global features thanks to the 
$ \phi $  mapping function, all to have a better segmentation. However, when 
K = 400, the CP value of the \emph{ERS} \cite{0781426508} and \emph{SEEDS} \cite{0781426509} algorithms are 0.151 
and 0.280 respectively, while the CP value of LSC is 0.366. Looking at the data 
in the table, this makes us understand how LSC is able to have a better BR 
value despite having a higher CP value than the previous algorithms, this 
represents the strength of this algorithm. 

\begin{figure}[h!]
    \centering
    \includegraphics[width = 1 \linewidth]{images/paper2/superpixelsComparison.png}
    \centering
    \caption{Superpixel results from different algorithms. (a) SEEDS. (b) NCuts. (c) SLIC. (d) ERS. (e) LSC.}
    \label{fig: superpixelsComparison}
\end{figure}

\subsection{APPLICATIONS}
\subsubsection{Class Segmentation}
An SVM-based multi-class object classifier trained on the histograms of the generated 
superpixels was used. In order for similar superpixels to have the same 
label, a conditional random field (CRF) is used to refine the segmentation. 
Thanks to this model, the prediction made by the classifier will also 
take into account the neighbors superpixels to the one to be labeled. The 
method used is that proposed in \cite{0781426534} where the main unit of measurement is 
that of the superpixel. In \cite{0781426534} the quick shift (\emph{QS}) algorithm is used for the 
generation of superpixels. The experiments were conducted on the \emph{Graz-02} 
database \cite{0781426535} containing already labeled objects. The accuracy achieved by 
the QS, ERS, SLIC and LSC algorithms is that shown in the Table \ref{table accuracy}, while 
the various segmentations, obtained from each of these, are visible in the 
figure \ref{fig: superpixelSegmentation}. As you can see, LSC performs a better segmentation than the other 
methods, this happens because the information obtained from the histograms 
belonging to the SIFT of different superpixel generated are used. In this way 
there is a better adherence to the boundary of each object in addition the 
superpixels take on a highly irregular shape for objects in the foreground and 
more regular for the background.
\begin{table}[h!]
    \centering
    \begin{adjustbox}{max width=\textwidth}
    \begin{tabular}{*{5}{|c}|}%%{|c|c|c|c|c|}
        \hline
        & QS & ERS & SLIC & LSC\\
        \hline
        bike & 72.2 & 74.2 & 76.3 & \bfseries{76.9}\\
        cars & 72.2 & 74.7 & 72.5 & \bfseries{76.8}\\
        person & 66.3 & 66.5 & 66.7 & \bfseries{67.0}\\
        \hline
    \end{tabular}
    \end{adjustbox}
    \caption{Accuracy using different superpixels algorithms.}
    \label{table accuracy}
\end{table}

\begin{figure}[htbp]
    \centering
    \includegraphics[width = 1 \linewidth]{images/paper2/superpixelAlgo.png}
    \centering
    \caption{Segmentation using different superpixels algorithms. (a) Original Image. (b) QS. (c) ERS. (d) SLIC. (e) LSC. (f)Ground Truth.}
    \label{fig: superpixelSegmentation}
\end{figure}

\subsubsection{Weakly Supervised Semantic Segmentation}
\begin{figure}[htbp]
    \centering
    \includegraphics[width = 0.4 \linewidth]{images/paper2/semanticSegmentation.png}
    \centering
    \caption{Weakly supervised semantic segmentation. (a) A training image with bounding boxes. (b) Output of soft-max layer of $ FCN_c $. (c) Coarse semantic segmentation result. (d) Superpixels with higher probability of foreground. (e) Fore-/Background segmentation result after iterative optimization. (f) Refined semantic segmentation result.}
    \label{fig: flowchartSemanticSegmentation}
\end{figure}
As the title suggests, the LSC algorithm is able to improve semantic segmentation 
with a weakly supervised method. The images of the PASCAL 
VOC2012 database \cite{0781426538} are used as tests, in which there are 1449 images
containing labeled objects and the remaining 10582 images containing only 
bounding boxes used for training and validation. The flow chart of the proposed 
method can be seen in figure (\ref{fig: flowchartSemanticSegmentation}). The steps are as follows: coarse 
semantic segmentation, fore-/background segmentation and refined semantic 
segmentation. In the first step there is a completely convolutional network, 
called $ FCN_c $, trained with images containing only the bounding boxes which 
will be restricted in order to eliminate the border pixels that seem to be 
irrelevant. These new squares are used as positive examples, while the cut 
parts are used as negative examples. For each superpixel \emph{p} generated within 
each bounding box $ i_{th} $, we set $ c_p^i $ as the average value of the colors of each 
pixel within the superpixel and then we set $ l_p^i $ as the label to be estimated (0 
backgroud, 1 foreground). The separation of the background from the 
foreground is seen as an optimization problem
\begin{equation}
    \begin{split}
        \argmax\limits_{l} \sum_i\sum_p(E_a(l_p^i,c_p^i) + \lambda_1E_c(l_p^i, FCN_c)\\
        + \lambda_2\sum_{q\in N(p)}E_s(l_p^i,c_p^i,l_q^i,c_q^i)) 
    \end{split}
\end{equation}
Where $ E_s $ is useful for capturing the smoothness prior produced by a covariance 
matrix, $ E_c $ represents the probability that the superpixel belongs 
to the foreground, while $ E_a $, also called the appearance model, represents 
the probability that a superpixel belongs to the background or foreground. 
After several iterations, which have the purpose of updating the foreground 
and background labels of the superpixels, a correct segmentation will be 
obtained. The segmentation obtained will be used to train a convolutional 
network called $ FCN_r $ which is combined with a dense CRF model useful for 
formulating the final semantic segmentation model. This system is also 
called $ Joint_{sp} $. The model is compared with other supervivided methods \cite{0781426541}, 
a strong one, called \emph{Strong}, and a weak one called \emph{Bbox-seg}, the results of 
which can be seen in figure \ref{fig: semanticSegmentation}.
\begin{figure}[htbp]
    \centering
    \includegraphics[width = 0.6 \linewidth]{images/paper2/segmentationAlgo.png}
    \centering
    \caption{Semantic segmentation. (a) Input image. (b) Strong. (c) Bbox-seg. (d) $ Joint_{sp} $ (LSC) }
    \label{fig: semanticSegmentation}
\end{figure}

While the accuracy, measured with the intersection over-union (\emph{IOU}), 
achieved by the systems mentioned is visible in the table \ref{table accuracy semantic segmentation}.
\begin{table}[h!]
    \centering
    \begin{adjustbox}{max width=\textwidth}
    \begin{tabular}{*{4}{|c}|}%%{|c|c|c|c|}
        \hline
        Strong & Bbox-seg & $ Joint_{sp} $ \\
        \hline
        62.5 & 60.6 & \bfseries{64.0} \\
        \hline
    \end{tabular}
    \end{adjustbox}
    \caption{Semantic segmentation accuracy in terms of Mean IOU (\%)}
    \label{table accuracy semantic segmentation}
\end{table}

%\newpage
%\section{An End-to-End Compression Framework Based
on Convolutional Neural Networks}

\begin{flushleft}
    \author{
    Feng Jiang, 
    Wen Tao, 
    Shaohui Liu, 
    Jie Ren, 
    Xun Guo, 
    Debin Zhao, 
    \emph{Member, IEEE}
    }
\end{flushleft}

\begin{center}
    \emph{IEEE TRANSACTIONS ON CIRCUIT AND SYSTEMS FOR VIDEO THECNOLOGY, VOL. 28, NO. 11, OCTOBER 2018}
\end{center}

\subsection{INTRODUCTION}

%\newpage
%\section{A Data Set for Camera-Independent Color Constancy}

\begin{flushleft}
    \author{
    Ça$ \breve{g} $lar Aytekin, 
    Jarno Nikkanen, 
    Moncef Gabbouj
    \emph{Fellow, IEEE}
    }
\end{flushleft}

\begin{center}
    \emph{IEEE TRANSACTIONS ON IMAGE PROCESSING, VOL. 27, NO. 2, FEBRUARY 2018}
\end{center}

\subsection{INTRODUCTION}
Color constancy is a characteristic of the human visual system (HVS) that 
helps to perceive a constant color, for example of an object, at different levels 
of illuminations. It is claimed that the achievement of constancy occurs by 
approximating the composition of the lighting in order to obtain the true 
color of the object. There are supervised and unsupervised methods 
that calculate color consistency. Unsupervised methods are divided into two 
categories, based on the techniques they use to estimate the color of 
the illuminating source. The first category makes statistical assumptions 
about reflectance in a scene. The second category instead uses the physical 
properties of objects in scenes. Supervised methods also fall into two categories. 
The first category tries to learn a combination of unsupervised methods to 
estimate the illumination. The second category builds its own model for 
learning about illumination. The major factor affecting all of these methods 
is the sensitivity of the camera sensor. When the sets used for training, 
validation and testing contain images taken by different cameras, the results 
returned by the algorithms may be different, as well as their performance. 
On the other hand, one method returns "fixed" results when operating on 
images from the same camera. In the CC field, this problem is called camera-independence.
This report provides a dataset, called Intel-TUT, which is useful for testing 
camera-independence in the CC. Three different cameras capture real scenes both in the lab and elsewhere. Laboratory images have different lighting conditions. The dataset contains 1536 images and a test set consisting of 454 images taken by a single camera.

%\newpage
%\section{An End-to-End Multi-Task and Fusion CNN for Inertial-Based Gait Recognition}

\begin{flushleft}
    \author{
    Rubén Delgrado-Esca$ \tilde{n} $o,
    Francisco M. Castro,
    Julián Ramos Cózar,
    Manuel J. Marín-Jiménez,
    and Nicolás Guil
    }
\end{flushleft}

\begin{center}
    \emph{IEEE DIGITAL OBJECT IDENTIFIER}
\end{center}

\subsection{INTRODUCTION}
An individual's gait appears as their own fingerprint. The study of this 
topic doesn't not only concern the area of medicine, but also that of security 
and identification. The gating study is based on a non-invasive system that 
collects patterns without the direct intervention of the subject. The data, 
which allow the recognition of the gait, are taken by inertial sensors present 
in a multitude of devices such as smartphones or smartwatches. A system 
is presented that uses a convolutional neural network (CNN) that uses the 
data collected by these inertial sensors to be able to predict the subject. The 
approach followed is the one in figure \ref{fig:preview}. Two extensions have been added 
to the network. The first deals with merging sensory data, while the second 
deals with improving the learning process and producing multiple outputs 
from a single input, thanks to a multi-task scheme. The tasks considered 
are: identification, gender recognition and age estimation. The dataset used 
is OU-ISIR gait database \cite{0857651721}, containing information such as inertial sensors 
such as accelerometers and gyroscopes. Gait recognition can be used in two contexts:
\begin{enumerate}
    \item {\bfseries{Identification}}: gait is used to obtain the identity of a known subject.
    \item {\bfseries{Authentication}}: gait is used to validate the identity of a known subject.
\end{enumerate}
\begin{figure}[htbp]
    \centering
    \includegraphics[width = 0.6 \linewidth]{images/paper5/usecase.png}
    \centering
    \caption{A preview of how the model works.}
    \label{fig:preview}
\end{figure}

\subsection{RELATED WORK}
Some methods have placed inertial sensors in different parts of the body 
such as legs \cite{0857651733}, hips \cite{0857651732}, ankles, or even in objects such as bags \cite{0857651735} and 
pockets \cite{0857651720}. Other methods \cite{0857651736} directly use all the inertial sensors present 
in the smartphone. With the advent of deep learning and CNNs, classifying 
an activity has turned out to be an easier job. The input provided to these 
networks is either the raw data of the inertial sensors, or the images that 
represented such data. The entire sequence of data is divided into segments, 
or windows, in order not to negatively affect the performance of the network. 
In the following work, a union of all the data produced by inertial sensors is 
carried out with the multi-task approach in order to provide a single model 
that uses different inputs to produce different outputs.

\subsection{PROPOSED APPROACH}
\subsubsection{Problem definition}
The proposed neural network is able to automatically extract the discriminat 
features from a gait sequence. There is no data pre-processing phase. The 
dataset used contains three labels for each fetature, indicating the age, years 
and gender of the subject, but in addition to this, any type of dataset with 
labels and sensory data would have been fine. In the following paper the 
following nomenclature will be used:
\begin{itemize}
    \item {\bfseries{\emph{S}}}: input temporal sequence of \emph{D} channel measurements taken by a sensor.
    \item {\bfseries{$ s_i $}}: sub-sequence of \emph{S} having length \emph{L}, given as input to CNN.
    \item $ y_i^t $: label of the sub-sequence $ s_i $ and taskt \emph{t}.
    \item $ g(s_i, \theta ) $: non-linear function applied to $ s_i $ with a set of parameters $ \theta $.
    \item {\bfseries{$ \hat{y_i} $}}: output of the network for a given $ s_i $ input.
\end{itemize}

\subsubsection{Initial CNN Architecture}
The length of the input is normalized before it can end up on CNN. Each 
sequence S is divided into U sub-sequences $ s_i $, where $ 1 \leq i \leq U $. However, 
a sequence is divided into windows having length L, with an overlap of 0\%, 
where each signal size defines a specific input channel. The number D of 
channels is useful for defining the 1 x N x D dimension of the filter of the 
first layer, where N indicates the dimension of the convolution. The proposed 
convolutional network is composed of 4 convolutional layers and a gradually 
increasing number of filters. After convolution there are elements such as 
ReLU, batch normalization (useful for faster learning) and max pooling. After 
the last convolutional layer, the average is chosen as a pooling operation. 
Then dropout and fully-connected layers are added with a number of outputs 
equal to the required classes. The last layer is composed of the Softmax 
normalization function which returns the probability distribution of each class. 
To train the model, only for the identification task, a cross-entropy is used 
as a loss function.
\begin{figure}[htbp]
    \centering
    \includegraphics[width = 1 \linewidth]{images/paper5/architecture.png}
    \centering
    \caption{Pipeline for multi-task and multi-sensor learning in a gait-based system}
    \label{fig:pipeline}
\end{figure}

\subsubsection{Multi-task Approach}
The aim is to training a deep multi-task model (DTM). In order to do this, 
there must be a set of tuples $ I = (s_i, y_i^m, y_1^1, ..., y_i^T) $ where $ y_i^m $ is the label 
of the main task, $ y_i^t $, with $ t \in [1,T] $, is the label for each auxilary task. 
Each individual task has its own loss function \emph{L}. The loss function (\ref{loss-function}) is 
calculated on each sub-sequence \emph{s} and is formed by the product of each single 
loss function, of each task, with the weights ($ \lambda $)  associated at each task.
\begin{eqnarray}\label{loss-function}
    \emph{$ L_{DTM} (g(s,\theta), Y) $} & = & \lambda_{id}L_{id}(\hat{y}^{id}, y^{id}) \nonumber \\
                                        &   & + \lambda_{age}L_{age}(\hat{y}^{age}, y^{age}) \nonumber \\
                                        &   & + \lambda_{gender}L_{gender}(\hat{y}^{gender}, y^{gender}) 
\end{eqnarray}
Where $ Y = (y^{id}_i, y^{age}_i, y^{gender}_i) $ and $ L_{id}, L_{age}, L_{gender} $ are the loss functions of 
each tasks. As for the identity and gender tasks, their loss function is the 
cross-entropy and is calculated with the formula \ref{cross-entropy}.
\begin{equation}\label{cross-entropy}
    \emph{$ L_m(\hat{y}, c) $} = -\hat{y}_c+\log\sum^K_{k=1}e^{\hat{y}k} 
\end{equation}
Where $ \hat{y} $ is the ouput vector of the network, $ \hat{y}_c $ is the output for target class, 
$ \hat{y}_k $ is the \emph{k}-th component of the output vector, \emph{c} is the ground-truth class 
and \emph{K} is the total number of classes.

\subsubsection{Modality Fusion}
The merging of data that occurs at the input is useful for increasing the accuracy of the 
network and allows understanding the relationships between 
the different types of input data. Each data coming from a specific sensor 
will be inserted in an individual branch, composed of a specific number of 
convolutional layers that will calculate specific predictors for each sensor. 
In the end, the data will be concatenated in a common branch which has the 
task of extracting the combined features from all the sensors. Based on the accuracy metric, the 
layer that will return the best result will be selected.

\subsubsection{Identity Authentication}\label{IA}
An authentication process begins by submitting a sample of input to CNN. 
Within this, there may be a layer that will produce a vector of features. 
To find out which layer it is, a cross-validation process is performed. Once 
extracted, the features are normalized with the L2 standard. Subsequently 
a vector is calculated containing the Euclidean distances between the values 
given in input and those present in the training set. In order to calculate the 
\emph{Area Under Curve} (AUC) or the \emph{Equal-Error-Rate} (EER), these distances 
are transformed into probabilities.

\subsection{EXPERIMENTS AND RESULTS}
\subsubsection{Dataset}
The OU-ISIR dataset is composed in two parts. Part A contains data on 744 
individuals, each having two sub-sequence (s) recording at a rate of 100Hz 
via an IMU sensor located at the waist of each. The first sequence is used for 
training, the second for testing. The labels associated with each individual 
are: identity, gender and age (split in ranges). Part B, on the other hand, 
is composed of 495 individuals with 3 IMU sensors located on the body. For 
each individual and sensor, there are two sequences of walking levels, an up-slope 
walk sequence and the other down-slope walk. The data that will be 
merged are those coming from sensors such as accelerometer and gyroscope.

\subsubsection{Input data}
A data augumentation process is carried out to increase the amount of data 
available for training. In this way, instead of giving a single sequence S 
as input, three will be given, more precisely:
\begin{itemize}
    \item A Gaussian noise is added to the input signal;
    \item The sequence S is resized with a random value between 0.7 and 1.1;
    \item A sequence of 10 numbers is first interpolated to S and then the resulting sequence is sampled.
\end{itemize}
Each sequence (S) is divided into 100 sub-sequences ($ s_i $) (100Hz = 1 per 
second), a number enough to contain a walk, with overlap equal to 75\%.

\subsubsection{Implementation details}
First 100 epochs are performed in order to train the model, with the training 
set divided into training and validation. At the end of the training, in order to 
be able to fine-tune the parameters, a further 50 epochs are performed with 
the training set only. The optimization is carried out with the stochastic 
gradient descent (SGD) standard with a mini-batch of 128 samples, weigth 
decay of 0.0005 and a moment of 0.9. The total accuracy is obtained by 
combining the accuracies derived from the output of each sub-sequence ($ s_i $) 
of the input S. To obtain this, simply multiply each probability obtained 
from each of the sub-sequences with the following equation:
\begin{equation}
    P(S=c) = \prod_{i=1}^UP_i(s_i=c)
\end{equation}
Where U is the number of subsequences $ s_i $, P(s=c) is the probability of assign the identity \emph{c} to the person in sequence S and $ P_i(s_i=c) $ is the probability of assign the identity \emph{c} to teh person in subsequence $ s_i $.

\subsubsection{Gait Recognition experiments}
Different types of experiments on recognition, and their respective performances, in terms of accuracy and F1-measure, are presented below.
\begin{enumerate}
    \item \emph{Sensor position}: the following experiment show that using 3 different 
    sensors, the model has greater accuracy and a high value of F1-measure, 
    when using data from the sensor positioned on the left side of the body.
    \begin{table}[h!]
        \centering
        \begin{adjustbox}{max width=\textwidth}
        \begin{tabular}{*{3}{|c}|}%%{|c|c|c|}
            \hline
            IMU Position & Acc & F1-score \\
            \hline
            center & 92.3 & 90.8\\
            left & \bfseries{95.2} & \bfseries{94.0}\\
            right & 91.5 & 90.0\\
            \hline
        \end{tabular}
        \end{adjustbox}
        \caption{Accuracy and F1-score of three different sensors.}
        \label{table accuracy and F1}
    \end{table}

    \item \emph{Single task with individuals sensors} (1 Task, 1 Sensor): being a multi-task 
    network, a CNN network is created for each sensor in combination 
    with each label. As there are three types of labels and two sensors, a 
    total of 6 CNN networks are trained. From the results obtained from 
    the experiment, it can be seen that both sensors affect the model in an 
    almost similar way, in terms of accuracy and F1-Measure. These values 
    indicate that both sensors can be used in order to achive the purpose 
    of identifying the walk.
    \begin{table}[h!]
        \centering
        \begin{adjustbox}{max width=\textwidth}
        \begin{tabular}{|c||ccc|c||ccc|c|}
            \hline
                & \multicolumn{4}{c||}{Acc} & \multicolumn{4}{c|}{F1-Score} \\
            \hline
                Architecture & Id & Age & Gender & Avg & Id & Age & Gender & Avg\\
            \hline
                SingleTask Accelerometer & 89.7 & 91.0 & 94.8 & 91.8 & 87.6 & 91.3 & 94.5 & 91.2\\
                SingleTask Gyroscope& 89.1 & 89.1 & 94.4 & 90.9 & 87.5 & 89.7 & 94.4 & 90.5\\
            \hline 
        \end{tabular}
        \end{adjustbox}
        \caption{Accuracy and F1-score (1 Task, 1 Sensor).}
        \label{table accuracy and F1 (1 Task - 1 Sensor)}
    \end{table}

    \item \emph{Multi-task with individual sensors} (+ Task, 1 Sensor): in this experiment 
    more tasks are assigned to each of the sensors (3 tasks for 1 
    sensor) implying the creation of two CNN networks. This experiment 
    is useful to see if there are any performance improvements with respect 
    to the single assignment (defined in the previous point). The loss function 
    is that described in equation \ref{loss-function}. Comparing the results obtained 
    in table \ref{table accuracy and F1 (1 Task - 1 Sensor)} with table \ref{table accuracy and F1 (more Task - 1 Sensor)}, it can be seen that the following experiment led 
    to a slight improvement in performance.
    \begin{table}[h!]
        \centering
        \begin{adjustbox}{max width=\textwidth}
        \begin{tabular}{|c||ccc|c||ccc|c|}
            \hline
                & \multicolumn{4}{c||}{Acc} & \multicolumn{4}{c|}{F1-Score} \\
            \hline
                Architecture & Id & Age & Gender & Avg & Id & Age & Gender & Avg\\
            \hline
                MultiTask Accelerometer & 90.9 & 93.3 & 95.9 & 93.4 & 89.1 & 93.3 & 95.9 & 92.8\\
                MultiTask Gyroscope & 90.1 & 90.1 & 94.8 & 91.7 & 88.3 & 90.5 & 94.9 & 91.2\\
            \hline 
        \end{tabular}
        \end{adjustbox}
        \caption{Accuracy and F1-score (+ Task, 1 Sensor).}
        \label{table accuracy and F1 (more Task - 1 Sensor)}
    \end{table}

    \item \emph{Selection of the fusion position}: The fusion can take place in any convolutional 
    layer, called the common branch, of the network. For each 
    layer, a number of filters are applied which increases as the 
    depth of the network increases. As can be seen from table \ref{table accuracy and F1 fusion}, the best performances 
    are obtained in the fusion obtained in the first convolutional layer of 
    the network. Subsequent convolutional layers score lower because the 
    amount of information used for training is low.
    \begin{table}[h!]
        \centering
        \begin{adjustbox}{max width=\textwidth}
        \begin{tabular}{*{3}{|c}|}%%{|c|c|c|}
            \hline
            Position & Acc & F1-score\\
            \hline
            Conv1 & \bfseries{94.2} & \bfseries{92.9}\\
            Conv2 & 94.0 & 92.7\\
            Conv3 & 93.4 & 92.0\\
            Conv4 & 89.0 & 86.7\\
            Conv5 & 89.0 & 86.7\\
            \hline
        \end{tabular}
        \end{adjustbox}
        \caption{Accuracy and F1-score of fusion level experiment.}
        \label{table accuracy and F1 fusion}
    \end{table}

    \item \emph{Single Task with Fusion} (1 Task, + Sensor): in this experiment 3 CNNs 
    are used, one for each task, each making up a branch. In this case, the 
    combination of data from different sensors will be useful to be able to 
    determine only 1 task and therefore only one class. The results obtained 
    are better than the previous points, except for the gender class where 
    the MultiTask approach achieves a better score.
    \begin{table}[h!]
        \centering
        \begin{adjustbox}{max width=\textwidth}
        \begin{tabular}{|c||ccc|c||ccc|c|}
            \hline
                & \multicolumn{4}{c||}{Acc} & \multicolumn{4}{c|}{F1-Score} \\
            \hline
                Architecture & Id & Age & Gender & Avg & Id & Age & Gender & Avg\\
            \hline
                SingleTask Fusion & 94.2 & 95.0 & 95.6 & 94.9 & 93.5 & 95.0 & 95.6 & 94.7\\
            \hline 
        \end{tabular}
        \end{adjustbox}
        \caption{Accuracy and F1-score (1 Task, + Sensor).}
        \label{table accuracy and F1 (1 Task, + Sensor)}
    \end{table}

    \item \emph{Multi-task with Fusion} (+ Task, + Sensor): the last experiment consists in applying the fusion in a multitasking context within the network. In this case it will be necessary to train a CNN for all three existing tasks. From the results shown in table \ref{table accuracy and F1 (+ Task, + Sensor)}, this architecture manages to achieve better performance than the others mentioned above, this is because the model has much more information (labels and inputs) to describe people. The strength of this architecture is that it manages to return a result for each task.
    \begin{table}[h!]
        \centering
        \begin{adjustbox}{max width=\textwidth}
        \begin{tabular}{|c||ccc|c||ccc|c|}
            \hline
                & \multicolumn{4}{c||}{Acc} & \multicolumn{4}{c|}{F1-Score} \\
            \hline
                Architecture & Id & Age & Gender & Avg & Id & Age & Gender & Avg\\
            \hline
                MultiTask Fusion & \bfseries{94.8} & \bfseries{96.1} & \bfseries{97.7} & \bfseries{96.2} & \bfseries{93.8} & \bfseries{96.3} & \bfseries{97.7} & \bfseries{95.9}\\
            \hline 
        \end{tabular}
        \end{adjustbox}
        \caption{Accuracy and F1-score (+ Task, + Sensor).}
        \label{table accuracy and F1 (+ Task, + Sensor)}
    \end{table}
\end{enumerate}

\subsubsection{Authentication Experiments}
In paragraph \ref{IA} we said that to see if two different samples belong to the 
same subject, an identity authentication process must be carried out. In 
other words, it is necessary to evaluate the goodness of a model by calculating 
the AUC and EER index. To do this, one had to first calculate the 
Euclidean distances between the sample vector present in the test set with 
the sample vectors present in the training set. Subsequently, it was necessary 
to transform these distances into probabilities that will be represented 
on a ROC curve. Once this is done, for each model previously described, 
the values for AUC (the higher the better) and EER (the lower the better), 
calculated on the ROC curve, are shown in table \ref{Authentication}. From the results shown 
in the table, the multitask model obtains worse performance than the singletask 
model, this is caused by the fact that in the multitask model the labels 
to be managed are many more than the single label (id) to be managed in 
the singletask model.
\begin{table}[h!]
    \centering
    \begin{adjustbox}{max width=\textwidth}
    \begin{tabular}{|c|cc|}
        \hline
        Architecture & EER & AUC\\
        \hline
        SingleTask Accelerometer & 1.47 & 99.91\\
        SingleTask Gyroscope & 2.50 & 99.80\\
        \hline
        MultiTask Accelerometer & 1.61 & 99.90\\
        MultiTask Gyroscope & 2.85 & 99.72\\
        \hline
        SingleTask Fusion & \bfseries{1.14} & \bfseries{99.93}\\
        \hline
        MultiTask Fusion & 1.34 & 99.92\\
        \hline
    \end{tabular}
    \end{adjustbox}
    \caption{Authentication accuracy.}
    \label{Authentication}
\end{table}

\subsubsection{State of art comparison}
Comparing the proposed system with those already existing in the state of 
art, both for gait recognition (Tab.\ref{Gait comparison}) and authentication (Tab.\ref{Authentication comparison}), the 
performances outperforms those of the competitors.
\begin{table}[h!]
    \centering
    \begin{adjustbox}{max width=\textwidth}
    \begin{tabular}{|c|ccc|c|}
        \hline
        CNN & Id & Age & Gender & Avg\\
        \hline
        AE-GDI-CC & 61.0 & - & - & -\\
        Muaaz et al. & 63.5 & - & - & -\\
        Ngo et al. & 70.2 & - & - & -\\
        Wei et al. & 83.8 & - & - & - \\
        \hline
        MultiTask Fusion & \bfseries{94.8} & \bfseries{96.1} & \bfseries{97.7} & \bfseries{96.2}\\
        \hline
    \end{tabular}
    \end{adjustbox}
    \caption{Gait recognition comparison with some methods.}
    \label{Gait comparison}
\end{table}

\begin{table}[h!]
    \centering
    \begin{adjustbox}{max width=\textwidth}
    \begin{tabular}{|c|c|}
        \hline
        Approach & EER\\
        \hline
        Gafurov et al. & 15.8 \\
        Derawi et al. & 14.3 \\
        Rong et al. & 14.3 \\
        Ngo et al. & 13.5 \\
        Imp GDI + i-vector & 7.1 \\
        NC GDI + i-vector & 5.6 \\
        \hline
        SingleTask Fusion & \bfseries{1.1}\\
        \hline
    \end{tabular}
    \end{adjustbox}
    \caption{Authentication comparison with some methods.}
    \label{Authentication comparison}
\end{table}

\subsubsection{Execution time during test}
MultiTask models are the most time-consuming, but it is also true that they 
produce three outputs simultaneously (Fig. \ref{fig:time}).
\begin{figure}[htbp]
    \centering
    \includegraphics[width = 0.6 \linewidth]{images/paper5/time.png}
    \centering
    \caption{Execution time of the proposed models}
    \label{fig:time}
\end{figure}

\subsection{CONCLUSIONS}
As a final consideration, the model to be used, in the case of gait recognition, 
is the one that merges the input information and adopts a multitask 
approach. On the other hand, as regards the authentication problem, it has 
been seen that it is better to use a singletask model.

%\newpage
%\section{Coarse-to-Fine Semantic Segmentation From Image-Level Labels}

\begin{flushleft}
    \author{
    Longlong Jing,
    Yucheng Chen,
    Yingli Tan,
    \emph{Fellow, IEEE}
    }
\end{flushleft}

\begin{center}
    \emph{IEEE TRANSACTIONS ON IMAGE PROCESSING, VOL.29, 2020}
\end{center}

\subsection{INTRODUCTION}


%\newpage
%\section{Variational-Based Mixed Noise Removal With CNN Deep Learning Regularization}

\begin{flushleft}
    \author{
    Faqiang Wang,
    Haiyang Huang,
    Jun Liu
    }
\end{flushleft}

\begin{center}
    \emph{IEEE TRANSACTIONS ON IMAGE PROCESSING, VOL.29, 2020}
\end{center}

\subsection{INTRODUCTION}

%\newpage
%\section{Visual Object Tracking via Multi-Stream Deep Similarity Learning Networks}

\begin{center}
    \author{
    Kunpeng Li,
    \emph{Student Member, IEEE},
    Yu Kong,
    \emph{Member, IEEE},
    Yunf Fu,
    \emph{Fellow, IEEE}
    }
\end{center}

\begin{center}
    \emph{IEEE TRANSACTIONS ON IMAGE PROCESSING, VOL.29, 2020}
\end{center}

\subsection{INTRODUCTION}
The goal of visual tracking systems is to be able to obtain the position of 
the tracked target even in subsequent frames. The problems to be solved are 
those of occlusion, background clutter, illumination variations, deformation 
etc. The existing state-of-the-art models carry out training online exclusively. 
Some methods, based on pre-trained convolutional networks (CNN), track 
the object based on the background. In these networks, stochastic gradient 
descent is applied in order to update the entire network. However, these 
models appear to be too slow and do not work well in real-time. The following 
paper uses a model that compares similarities and is trained offline in order to 
predict the patch present in the next frame. The method, thanks to the use of 
the relative distance, is robust in the presence of phenomena that introduce 
the so called distractors. The purpose is to be able to compare the patch that 
identifies the target template (the object) with all the possible positions 
of the same region belonging to the next frame. In addition to tracking the 
object, the proposed model is a framework for EMDSLT tracking like the 
one shown in figure \ref{fig:EMDSLT}. Within it, two procedures are considered important: 
updating the model and self-recovery from failures. The network is composed 
of two types of structures, one that is responsible for dast-speed verification 
and production of the tracking results, while the second is responsible for re-verification 
and re-detection of the target object on the patches previously 
generated.
\begin{figure}[h!]
    \centering
    \includegraphics[width = \linewidth]{images/paper8/EMDSLT.png}
    \centering
    \caption{The framework of EMDSLT proposed method.}
    \label{fig:EMDSLT}
\end{figure}.

%\newpage
%\section{Similar Face Recognition Using the IE-CNN Model}

\begin{center}
    \author{
    An-Ping Song,
    Qian Hu,
    Xue-Hai Ding,
    Xin-Yi Di,
    Zi-Heng Song
    }
\end{center}

\begin{center}
    \emph{DIGITAL OBJECT IDENTIFIER}
\end{center}

\subsection{INTRODUCTION}

\newpage
\section{Convolutional Neural Networks for Risso’s Dolphins Identification}

\begin{center}
    \author{
    Rosalia Maglietta,
    Vito Renò,
    Rocco Cacioppoli,
    Emanuele Seller,
    Stefano Bellomo,
    Francesca Cornelia Santacesaria,
    Roberto Colella,
    Giulia Cipriano,
    Ettore Stella,
    Karin Hartman,
    Carmelo Fanizza,
    Giovanni Dimauro,
    \emph{(Member, IEEE)},
    Roberto Carlucci
    }
\end{center}

\begin{center}
    \emph{DIGITAL OBJECT IDENTIFIER}
\end{center}

\subsection{INTRODUCTION}
Photo identification is a technique that can be used on marine species in 
order to safeguard the environment and the various existing species. The 
marine species on which this study is based are cetaceans, in particular the 
Risso's dolphin species Grampus Griseus (Fig. \ref{fig:Risso}). This species is the least 
known and is present in greater numbers in Mediterranean waters. In order to 
recognize a dolphin, the scars present on their body and on the dorsal fin are 
used, representing the "fingerprints" useful for identifying them. To achieve 
this, the RUSPool algorithm is used which aims to perform an intelligent 
merge, through a custom filter, of the m RUSBoost classifiers. Subsequently, 
a new methodology called Neural Network Pool (NNPool) will be used, also 
useful for recognizing only the Risso species of dolphins. This algorithm is 
composed of a pool of CNNs useful for achieving the final goal. The chosen 
output will be dictated by the major voting of all these networks. Finally, 
the results obtained will be compared with those obtained by the RusPool 
algorithm, used for the same purpose.
\begin{figure}[h!]
    \centering
    \includegraphics[width = 0.8\linewidth]{images/paper10/Risso.png}
    \centering
    \caption{Risso’s dolphins Grampus Griseus}
    \label{fig:Risso}
\end{figure}

\newpage
\bibliographystyle{abbrv}
\bibliography{Bibliography}

\end{document}