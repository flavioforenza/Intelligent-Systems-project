\section{Linear Spectral Clustering Superpixel}

\begin{flushleft}
    \author{
    Jiansheng Chen, 
    \emph{Member, IEEE}, 
    Zhengqin Li,
    \emph{Student Member, IEEE}, 
    Bo Huang 
}
\end{flushleft}

\begin{center}
    \emph{IEEE TRANSACTIONS ON IMAGE PROCESSING, VOL. 26, NO. 7, JULY 2017}
\end{center}

\subsection{INTRODUCTION}
The introduced technique is called SUPERPIXEL. Widely used in image 
processing for particular tasks such as image segmentation, image analysis, 
image classification, target tracking, 3D reconstruction, surface retrieval and 
object proposal. The purpose of this technique is to be able to group the 
pixels into groups that delimit the edges of an object in order to be able 
to extract its content. Compared to other methods already existing in the 
state of the art, characteristics such as size, number of superpixels and shape 
are not considered. The purpose of the elaborate story is to reduce the 
computational complexity. The three targets that must be satisfied by each 
superpixel algorithm are:
\begin{enumerate}
    \item Adhere well to the edges without forming overlaps on objects;
    \item Have a pre-processing technique useful to improve efficiency;
    \item Consider global information.
\end{enumerate}
The proposed system, called Linear Spectral Clustering (LSC), manages to 
satisfy all the previous points with a high memory efficiency. In LSC, each 
pixel is mapped to a point within a ten-dimensional space of characteristics 
in which the weighted K-means is applied for segmentation. 

\subsection{LSC SUPERPIXEL}
The study focuses on the relationships between the results returned by the 
optimization functions, and those returned by two other equations. If the results 
are equivalent, then weighted k-means clustering can replace the highly 
complex eigen based method.

\subsubsection{Mathematical Backgrounds}
The task of LSC is to discover the relationships that exist between the objective 
functions ($ F_{N_{cuts}} $) (\ref{FNcuts}), of the normalized cuts, and the objective functions 
of the weighted K-means ($ F_{km}$) (\ref{Fkm}). 
\begin{equation} \label{Fkm}
    F_{km} = \sum_{k=1}^K\sum_{p\in\pi_k}\omega(p)= || \phi(p)-m_k ||^2 
\end{equation}
\begin{equation} \label{Centroid}
    m_k = \frac{\sum_{q\in\pi_k}\omega(q)\phi(q)}{\sum_{q\in\pi_k}\omega(q)}
\end{equation}
\begin{equation} \label{FNcuts}
    F_{N_{cuts}} = \frac{1}{K}\sum_{k=1}^K\frac{\sum_{p\in\pi_k}\sum_{q\in\pi_k}W(p,q)}{\sum_{p\in\pi_k}\sum_{q\in{V}}W(p,q)}
\end{equation}
Where $ m_k $ (\ref{Centroid}) represent the average point, or centroid, of each cluster. Within 
the formulation of normalized cuts, each data Point corresponds to a node 
within a graph \emph{G = (V, E, W)} where \emph{V} is the set of all nodes, \emph{E} represents 
the set of all edges connected and \emph{W} represents the result returned by a 
similarity function between nodes. The criterion of normalized cuts is based 
on maximizing the $ F_{N_{cuts}} $ function. The minimization or maximization problems 
are called \emph{optimization problems} which can be solved by using a positive kernel of 
a matrix. To see the relationship between the two functions, the Dhillon's concept \cite{0781426526} is extended, to obtain {\bfseries Corollary \ref{Corollary1}}:

\begin{corollary} \label{Corollary1}
    Optimizations of the objective functions of the weighted K-means
    and the normalized cuts are mathematically equivalent if (\ref{eq4}) and (\ref{eq5}) 
    hold simultaneously. The symbol $ \cdot $ stands for inner product.
\end{corollary}

\begin{equation} \label{eq4}
    \omega(p)\pi(p) \cdot \omega(q)\pi(q) = W(p,q), \forall p,q \in V
\end{equation}

\begin{equation} \label{eq5}
    w(p) = \sum_{q \in V} W(p,q), \forall p \in V
\end{equation}

After making various derivative calculations, $ F_{km} $ can be seen as: 
\begin{equation}
    F_{km} = C - K * F_{N_{cuts}} 
\end{equation}

From the above definition it is possible to note that the minimization of 
$ F_{km} $ is equivalent to the maximization of $ F_{N_{cuts}} $. Specifically, both $ F_{km} $ and 
$ F_{km} $ perform an identical partitioning of the n-dimensional space defined by 
the function $ \phi $.

\subsubsection{LSC Algorithm}

