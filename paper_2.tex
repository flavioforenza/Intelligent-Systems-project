\section{Linear Spectral Clustering Superpixel}

\begin{flushleft}
    \author{
    Jiansheng Chen, 
    \emph{Member, IEEE}, 
    Zhengqin Li,
    \emph{Student Member, IEEE}, 
    Bo Huang 
}
\end{flushleft}

\begin{center}
    \emph{IEEE TRANSACTIONS ON IMAGE PROCESSING, VOL. 26, NO. 7, JULY 2017}
\end{center}

\subsection{INTRODUCTION}
The introduced technique is called SUPERPIXEL. Widely used in image 
processing for particular tasks such as image segmentation, image analysis, 
image classification, target tracking, 3D reconstruction, surface retrieval and 
object proposal. The purpose of this technique is to be able to group the 
pixels into groups that delimit the edges of an object in order to be able 
to extract its content. Compared to other methods already existing in the 
state of the art, characteristics such as size, number of superpixels and shape 
are not considered. The purpose of the elaborate story is to reduce the 
computational complexity. The three targets that must be satisfied by each 
superpixel algorithm are:
\begin{enumerate}
    \item Adhere well to the edges without forming overlaps on objects;
    \item Have a pre-processing technique useful to improve efficiency;
    \item Consider global information.
\end{enumerate}
The proposed system, called Linear Spectral Clustering (LSC), manages to 
satisfy all the previous points with a high memory efficiency. In LSC, each 
pixel is mapped to a point within a ten-dimensional space of characteristics 
in which the weighted K-means is applied for segmentation. 

