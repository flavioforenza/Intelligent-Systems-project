\section{An End-to-End Compression Framework Based
on Convolutional Neural Networks}

\begin{flushleft}
    \author{
    Feng Jiang, 
    Wen Tao, 
    Shaohui Liu, 
    Jie Ren, 
    Xun Guo, 
    Debin Zhao, 
    \emph{Member, IEEE}
    }
\end{flushleft}

\begin{center}
    \emph{IEEE TRANSACTIONS ON CIRCUIT AND SYSTEMS FOR VIDEO THECNOLOGY, VOL. 28, NO. 11, OCTOBER 2018}
\end{center}

\subsection{INTRODUCTION}
In recent years, within the field of computer vision, remarkable results have 
been achieved with regard to image compression. The purpose of compression 
is to be able to transmit, or save, the entire image at low bit rates. As far as 
decompression is concerned, deblocking and denoising techniques have been 
developed that are useful for obtaining good images. Further pre-processing 
steps have been found to negatively impact in system performance. Therefore 
the proposed method uses a framework composed of two convolutional 
networks (\emph{CNNs}). The first network, called compact convolutional neural 
newtork (\emph{comCNN}), is used for compression and uses the JPEG, JPEG2000 
and BPG encoding codecs. The second network, called reconstruction convolutional 
neural network (\emph{RecCNN}), is used for image decompression. A 
first example of the work carried out by the proposed framework is visible in 
the Fig. \ref{fig:output}.
\begin{figure}[h!]
    \centering
    \includegraphics[width = 0.6 \linewidth]{images/paper3/output .png}
    \centering
    \caption{Left: the JPEG-coded image. Right: the decoded image.}
    \label{fig:output}
\end{figure}