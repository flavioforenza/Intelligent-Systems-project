\section{A Data Set for Camera-Independent Color Constancy}

\begin{flushleft}
    \author{
    Ça$ \breve{g} $lar Aytekin, 
    Jarno Nikkanen, 
    Moncef Gabbouj
    \emph{Fellow, IEEE}
    }
\end{flushleft}

\begin{center}
    \emph{IEEE TRANSACTIONS ON IMAGE PROCESSING, VOL. 27, NO. 2, FEBRUARY 2018}
\end{center}

\subsection{INTRODUCTION}
Color constancy is a characteristic of the human visual system (HVS) that 
helps to perceive a constant color, for example of an object, at different levels 
of illuminations. It is claimed that the achievement of constancy occurs by 
approximating the composition of the lighting in order to obtain the true 
color of the object. There are supervised and unsupervised methods 
that calculate color consistency. Unsupervised methods are divided into two 
categories, based on the techniques they use to estimate the color of 
the illuminating source. The first category makes statistical assumptions 
about reflectance in a scene. The second category instead uses the physical 
properties of objects in scenes. Supervised methods also fall into two categories. 
The first category tries to learn a combination of unsupervised methods to 
estimate the illumination. The second category builds its own model for 
learning about illumination. The major factor affecting all of these methods 
is the sensitivity of the camera sensor. When the sets used for training, 
validation and testing contain images taken by different cameras, the results 
returned by the algorithms may be different, as well as their performance. 
On the other hand, one method returns "fixed" results when operating on 
images from the same camera. In the CC field, this problem is called camera-independence.
This report provides a dataset, called \emph{Intel-TUT}, which is 
useful for testing camera-independence in the CC. Three different cameras 
capture real scenes both in the lab and elsewhere. Laboratory images have 
different lighting conditions. The dataset contains 1536 images and a test 
set consisting of 454 images taken by a single camera.

\subsection{COLOR CONSTANCY DATASETS}
One of the first datasets for calculating the CC was the one proposed in 
\cite{0807099130}. The single camera captured images showing a total of 1995 surfaces, 
11 lab illuminations and 81 illuminations from the real world. The dataset 
also contains a number of images that make up 50 scenes. After a correct 
calibration, it removes the irrelevant images, the remainder was made up of 
only 529 images which made up 30 scenes. Each of these images belongs to 
the set captured in the lab. Another dataset is the one proposed in \cite{0807099132}, consisting 
of 11,000 images of indoor / outdoor scenes. The scenes were captured in 
different geographic locations and under different weather conditions. Unfortunately 
this dataset contains low resolution images that require a correction 
phase. Another dataset containing 246 indoor images and 322 outdoor images, 
taken by two different cameras, is the one proposed in \cite{0807099120}. Another 
dataset that uses the auto-bracketing technique to acquire the images is the 
one proposed in \cite{0807099134}. In this dataset they are acquired up to 9 different 
images, of the same scene, with different settings for each shot. A dataset that 
aims to obtain multiple lighting in a scene is the one proposed in \cite{0807099135}. 
This contains relatively few images, 9 acquired in an outdoor environment 
and 59 lab images. The is also a dataset containing the videos mentioned 
in \cite{0807099136}. There was only one study \cite{0807099129} that relies on color conversion in order 
to achieve camera independence. However, this method requires a very 
sensitive spectral camera to achieve good performance, so it is not applicable 
to all images. Finally, a last dataset consisting of 1600 indoor and outdoor 
images, taken by 9 different cameras, is the one proposed in \cite{0807099128} called NUS. 
In this collection, each scene was captured by each camera with slight misalignments. 
The database proposed in this paper is similar to the NUT only 
for the different cameras used. However, the proposed database has some 
features such as the following:
\begin{itemize}
    \item Provides the spectral sensitivities of the cameras;
    \item Different illuminants illuminate the scene;
    \item Among the different cameras used, one is mobile;
    \item Provides a set of tests for good evaluation of CC methods.
\end{itemize}

\subsection{THE PROPOSED INTEL-TUT DATA SET}
The purpose of the dataset is to help the algorithms to determine if the color 
constancy value obtained is to be considered good or not. If an algorithm 
does not find a change in this value on the same set of images taken by the 
same camera, then this algorithm is able to have a uniform color constancy 
and consequently would produce optimal outputs.

\subsubsection{Light Sources}
A variety of devices were used that were capable of producing a light source. 
Each property such as luminance (Lux), color temperature (CCT) and CIE 
xy chromaticity, have been appropriately set to have an equitable acquisition.

\subsubsection{Cameras}
Three types of camera are used for the construction of the dataset. Two 
of these belong to the high-end, while a third belongs to the category of 
mobile phones. To achieve correct color correction, several 3x3 size color 
conversion matrices (\emph{CCMs}) were used. In other words, try to transform 
the components of the image from RGB to sRGB in order to adapt them to 
the lighting under which the image is viewed. For the outdoors, 10 CCMs 
were used based on the type of illumination. Another factor that required 
correction was the color shading (\emph{CS}). This only requires correction for the 
images coming from the mobile phone.