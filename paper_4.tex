\section{A Data Set for Camera-Independent Color Constancy}

\begin{flushleft}
    \author{
    Ça$ \breve{g} $lar Aytekin, 
    Jarno Nikkanen, 
    Moncef Gabbouj
    \emph{Fellow, IEEE}
    }
\end{flushleft}

\begin{center}
    \emph{IEEE TRANSACTIONS ON IMAGE PROCESSING, VOL. 27, NO. 2, FEBRUARY 2018}
\end{center}

\subsection{INTRODUCTION}
Color constancy is a characteristic of the human visual system (HVS) that 
helps to perceive a constant color, for example of an object, at different levels 
of illuminations. It is claimed that the achievement of constancy occurs by 
approximating the composition of the lighting in order to obtain the true 
color of the object. There are supervised and unsupervised methods 
that calculate color consistency. Unsupervised methods are divided into two 
categories, based on the techniques they use to estimate the color of 
the illuminating source. The first category makes statistical assumptions 
about reflectance in a scene. The second category instead uses the physical 
properties of objects in scenes. Supervised methods also fall into two categories. 
The first category tries to learn a combination of unsupervised methods to 
estimate the illumination. The second category builds its own model for 
learning about illumination. The major factor affecting all of these methods 
is the sensitivity of the camera sensor. When the sets used for training, 
validation and testing contain images taken by different cameras, the results 
returned by the algorithms may be different, as well as their performance. 
On the other hand, one method returns "fixed" results when operating on 
images from the same camera. In the CC field, this problem is called camera-independence.
This report provides a dataset, called Intel-TUT, which is useful for testing 
camera-independence in the CC. Three different cameras capture real scenes both in the lab and elsewhere. Laboratory images have different lighting conditions. The dataset contains 1536 images and a test set consisting of 454 images taken by a single camera.