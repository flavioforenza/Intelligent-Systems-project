\section{Convolutional Neural Networks for Risso’s Dolphins Identification}

\begin{center}
    \author{
    Rosalia Maglietta,
    Vito Renò,
    Rocco Cacioppoli,
    Emanuele Seller,
    Stefano Bellomo,
    Francesca Cornelia Santacesaria,
    Roberto Colella,
    Giulia Cipriano,
    Ettore Stella,
    Karin Hartman,
    Carmelo Fanizza,
    Giovanni Dimauro,
    \emph{(Member, IEEE)},
    Roberto Carlucci
    }
\end{center}

\begin{center}
    \emph{DIGITAL OBJECT IDENTIFIER}
\end{center}

\subsection{INTRODUCTION}
Photo identification is a technique that can be used on marine species in 
order to safeguard the environment and the various existing species. The 
marine species on which this study is based are cetaceans, in particular the 
Risso's dolphin species Grampus Griseus (Fig. \ref{fig:Risso}). This species is the least 
known and is present in greater numbers in Mediterranean waters. In order to 
recognize a dolphin, the scars present on their body and on the dorsal fin are 
used, representing the "fingerprints" useful for identifying them. To achieve 
this, the RUSPool algorithm is used which aims to perform an intelligent 
merge, through a custom filter, of the m RUSBoost classifiers. Subsequently, 
a new methodology called Neural Network Pool (NNPool) will be used, also 
useful for recognizing only the Risso species of dolphins. This algorithm is 
composed of a pool of CNNs useful for achieving the final goal. The chosen 
output will be dictated by the major voting of all these networks. Finally, 
the results obtained will be compared with those obtained by the RusPool 
algorithm, used for the same purpose.
\begin{figure}[h!]
    \centering
    \includegraphics[width = 0.8\linewidth]{images/paper10/Risso.png}
    \centering
    \caption{Risso’s dolphins Grampus Griseus}
    \label{fig:Risso}
\end{figure}

\subsection{MATERIAL AND METHODS}
\subsubsection{Survey Area and Data Collection}
A dataset was created containing all the sightings of dolphins that took place 
in the years between 2013 and 2018, in the Gulf of Taranto. The collection 
contains a total of 93 different labeled Risso dolphins, but sadly only 23 of 
these were selected as each has at least 15 images on each side of the fin. 
Through the use of an $f_R$ mask filter, 70\% of each image consists of the 
dolphin fin alone, excluding external elements such as the sea. This filter 
led to consider a new dataset containing 433 images belonging to the 23 
dolphins. The $D_{NN}$ dataset used to train CNNs contains images that have 
not been pre-processed by the $f_R$ filter. This latest dataset is made up of 
582 images belonging to 28 different dolphins. A validation dataset was also 
created containing 300 images of known and unknown Risso's dolphins.

\subsubsection{RUSBoost Methodology}
The RUSBoost algorithm has the task of identifying each dolphin and predicting 
its label. The application of this algorithm can be schematized in 
figure \ref{fig:RUSBoost}. The architecture consists of 3 main phases:
\begin{enumerate}
    \item \emph{Pre-processing}: creation of the black and white image containing the 
    extracted fin.
    \item \emph{Features Extraction}: build the 3-dimensional super-SIFT descriptor. 
    \item \emph{Classification}: application of the RUSBoost algorithm.
\end{enumerate}
RUSBoost consists of two other types of algorithms: Random Under-Sampling 
(RUS) and Adaboost \cite{0907875811}. RUSBoost training takes place by carrying out 
cross-validation, one-vs-all, on a set of labeled data of the 23 dolphins considered.
\begin{figure}[h!]
    \centering
    \includegraphics[width = 0.8\linewidth]{images/paper10/RUSBoost.png}
    \centering
    \caption{Experimental methodology based on RUSBoost.}
    \label{fig:RUSBoost}
\end{figure}

\subsubsection{RUSPool}
