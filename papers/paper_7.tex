\section{Variational-Based Mixed Noise Removal With CNN Deep Learning Regularization}

\begin{center}
    \author{
    Faqiang Wang,
    Haiyang Huang,
    Jun Liu
    }
\end{center}

\begin{center}
    \emph{IEEE TRANSACTIONS ON IMAGE PROCESSING, VOL.29, 2020}
\end{center}

\subsection{INTRODUCTION}
Random noise distributions correspond to standard probabilistic distributions, 
such as the Gaussian, Poisson and other distributions. There have 
been many methods that attempt to clean up noisy images. The classical 
methods are based on the use of filters such as Gaussian, Gabor or median 
filters. Other tools such as the Variational method have been widely used. 
This method is based on reducing the cost function containing the fidelity 
term of the data, useful for calculating the difference between true and observed 
data, and the regularization terms. Another method used was that of 
TV regularization. Unfortunately this method fails to preserve texture detail. 
Nonlocal methods, on the other hand, are able to capture image details. 
These methods exploit the self-similarity properties that can be integrated 
into variational methods. Despite these methods, the function of the weights 
is difficult to determine. Methods based on deep learning have achieved good 
noise reduction performance. Many of these assume that the value can be 
removed by applying the L2 norm. According to the study carried out in 
the following article, in order to remove the mixed noise it is necessary to 
determine the type and level of noise in each pixel. The methods based on 
the variation, unlike the neural networks, require a single image and can 
integrate different techniques of regularization useful to be able to remove 
the single noise and the mixed noise. The complexity of these methods lies 
in being able to create a non-computationally complicated design. In the 
proposed article, the EM (Expectation-Maximization) algorithm is used to 
remove the noise, with the integration of a CNN process as regularization, 
in order to create a new variational method. The method is able to iteratively 
estimate the noise parameters useful for classifying the type and level 
of noise in each pixel. The work carried out is divided into four phases: 
regularization, synthesis, parameter estimation and error classification.

\subsection{RELATED WORK}
\subsubsection{Mixed Noise Model}
The mixed noise is:
\begin{equation}
    n = \left\{ \\
    \begin{array}{cc}
        n_1, & with~probability~r_1,\\
        n_2, & with~probability~r_2,\\
        \cdots, & \cdots\\
        n_k, & with~probability~r_k,\\ 
    \end{array}
     \right\}
\end{equation}
where $ n_k $ is the \emph{k}-th noise component with probability density function (PDF) 
$ p_k $ and $ r_k $ are the unknown mixture ratios and thei sum is equal to 1.

\subsubsection{Weighted K-SVD Model}
The following work is based on the method proposed in \cite{0884882814}. After obtaining 
the value of the mixed-noise \emph{Probability Density Function} (PDF), and 
assuming that the noise has a uniform distribution, we can derive the \emph{negative 
log-likelihood} function (\ref{PDF}) useful as {\bfseries{fidelity term}} in the variational 
method. 
\begin{equation}\label{PDF}
    L(u,\Theta) = - \sum_{x\in\Omega}\ln\sum_{k=1}^Kr_kp_k(u(x)-f(x))
\end{equation}
In order for this probability to be extended to more types of noise 
distribution, it is necessary to perform an optimization that minimizes the 
upper limit of the function (\ref{H function}). 
\begin{eqnarray}\label{H function}
    H(u,\Theta,w) & = & - \sum_{x\in\Omega}\sum_{k=1}\ln\left[r_kp_k(u(x)-f(x))\right]w_k \nonumber \\
                  &   & + \sum_{x\in\Omega}\sum_{k=1}w_k(x)\ln{w_k(x)}  
\end{eqnarray}
To minimize H, a minimization of $ w $ must be performed, which is a probability 
indicates the noise at each pixel, and an iterative minimization of the other 
two variables $ u $ and $ \Theta $.

\subsubsection{IRCNN}
Another method \cite{0884882819} uses a CCN, based on the IRCNN model, to restore an 
image in which Gaussian noise is present. In this model, patches based $ L^2 $ 
residual loss functions are adopted. Unfortunately, if there are mixed-noises, 
this method does not produce satisfactory results.

\subsection{THE PROPOSED METHOD (EM-CNN)}
Removal of mixed-noise can be done with the following model:
\begin{equation}
    (u^*, \Theta^*, w^*) = \argmin_x\left\{H(u,\Theta,w) + \lambda_1S(u)\right\}
\end{equation}
where $ S $ as regularizer, such \emph{Total Variation Regularization} (TV) while $ \lambda_1 $ 
control the balance of the terms. By changing $u$ with an auxiliary function 
$v$, and applying the \emph{augmented Lagrangian method} (ALM) \cite{0884882833}, the minimizations 
on each component of $H$ ($u$, $\Theta$, $w$) change. The resolution of these 
minimizations allows to divide the mixed-noise removal problem into Gaussian 
noise removal (renewing $u$), choice of the fidelity term (renewing $v$), 
noise reduction (updating the Lagragian multiplier $u$), parameter estimation 
noise ($\Theta$ update) and error classification ($w$ update). The formula used by 
the model is able to handle different types of Gaussian noises such as impulse 
noise, mixed-noise Gaussian-impulse and many others. Each noise has 
a certain probability function pk, so just change this value in (\ref{H function}). A CNN 
can be used to learn Gaussian noise for different types of levels. In this paper 
the CNN based denoiser \cite{0884882819} was used. The weight $w$ is useful for assigning 
different fidelity terms to pixels contaminated by different levels of types of 
noise. The minimization of $\Theta$ and $w$ are part of EM processes. Thanks to 
these it is possible to extract information such as noise variations and the 
classification of this in different classes.
