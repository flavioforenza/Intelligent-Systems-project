\section{Visual Object Tracking via Multi-Stream Deep Similarity Learning Networks}

\begin{center}
    \author{
    Kunpeng Li,
    \emph{Student Member, IEEE},
    Yu Kong,
    \emph{Member, IEEE},
    Yunf Fu,
    \emph{Fellow, IEEE}
    }
\end{center}

\begin{center}
    \emph{IEEE TRANSACTIONS ON IMAGE PROCESSING, VOL.29, 2020}
\end{center}

\subsection{INTRODUCTION}
The goal of visual tracking systems is to be able to obtain the position of 
the tracked target even in subsequent frames. The problems to be solved are 
those of occlusion, background clutter, illumination variations, deformation 
etc. The existing state-of-the-art models carry out training online exclusively. 
Some methods, based on pre-trained convolutional networks (CNN), track 
the object based on the background. In these networks, stochastic gradient 
descent is applied in order to update the entire network. However, these 
models appear to be too slow and do not work well in real-time. The following 
paper uses a model that compares similarities and is trained offline in order to 
predict the patch present in the next frame. The method, thanks to the use of 
the relative distance, is robust in the presence of phenomena that introduce 
the so called distractors. The purpose is to be able to compare the patch that 
identifies the target template (the object) with all the possible positions 
of the same region belonging to the next frame. In addition to tracking the 
object, the proposed model is a framework for EMDSLT tracking like the 
one shown in figure \ref{fig:EMDSLT}. Within it, two procedures are considered important: 
updating the model and self-recovery from failures. The network is composed 
of two types of structures, one that is responsible for dast-speed verification 
and production of the tracking results, while the second is responsible for re-verification 
and re-detection of the target object on the patches previously 
generated.
\begin{figure}[h!]
    \centering
    \includegraphics[width = \linewidth]{images/paper8/EMDSLT.png}
    \centering
    \caption{The framework of EMDSLT proposed method.}
    \label{fig:EMDSLT}
\end{figure}.